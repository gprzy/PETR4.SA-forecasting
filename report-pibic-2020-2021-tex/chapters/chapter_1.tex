\chapter{Introdução}

\pagenumbering{arabic}
\setcounter{page}{11}

\par
A previsão de movimentos do mercado financeiro é objeto de estudo de inúmeras abordagens, historicamente se adaptando para elevar ao máximo a precisão de seus modelos preditivos e inevitavelmente culminando na instrumentalização da tecnologia para o processamento e implementação de técnicas cada vez mais completas e precisas, diante da volatidade e o volume massivo de dados característicos deste mercado. \cite{senDattaChau2015}.

\par
Consequentemente, é evidente o crescimento da abordagem baseada nas Técnicas de Aprendizagem de Máquina (“\textit{Machine Learning}”) no que tange à análise de \textbf{Séries Temporais Financeiras}, especificamente mencionando as técnicas de \textit{Deep Learning}, notoriamente expressivas nos últimos cinco anos\footnote[1]{Disponível em: \href{https://scholar.google.com.br/scholar?q=\%22CNN\%22+\%22LSTM\%22+\%22bolsa+de+valores\%22\&hl=pt-BR\&as\_sdt=0\%2C5\&as\_ylo=2000\&as\_yhi=2021}{Google Acadêmico}. Acesso em 30 jan. 2021.}. Tais abordagens relacionadas à Séries Temporais Financeiras partem da análise do histórico passado de uma ação, e ações similares, realizando assim aproximações acerca de seu comportamento futuro; os padrões extraídos usualmente são combinados a técnicas específicas de mineração de dados, tais como associação, agrupamento, classificação, sumarização e regressão \cite{fu2011}. As Séries Temporais Financeiras são um subcampo de estudo de uma área matemática e estatística denominada de Séries Temporais, como é o caso das Séries provenientes da bolsa de valores: essencialmente dinâmicas, não lineares, complexas, não paramétricas e caóticas \cite{abuMostafaAtiya96}.

\par
Considerando a motivação dos investidores ao elaborar uma carteira, objetiva-se obter lucro máximo, em contrapartida de um risco mínimo; assim, torna-se evidente a problemática aqui destacada: diante da complexidade do problema, quais são as melhores formas de se utilizar as técnicas atuais de \textit{Deep Learning} na \textbf{classificação} e predição dos movimentos futuros da Bolsa de Valores? Em decorrência dos fatores acima mencionados, a emergência da nova abordagem, especialmente no que se refere às técnicas de \textit{Deep Learning}, considerando sua eficácia em inúmeros campos, contrastam com a ausência de trabalhos que busquem avaliar seu desempenho e identificar situações potenciais de uso, justificando assim a análise e avaliação a serem feitas nesse trabalho.