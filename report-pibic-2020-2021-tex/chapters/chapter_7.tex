\chapter{Considerações Finais}

\par
O presente trabalho avaliou o desempenho das redes neurais CNN e LSTM para as ações PETR4.SA (Petrobrás), para os anos de 2019 e 2020, comparando com o modelo Autorregressivo – clássico na previsão em séries temporais – bem como comparando a um modelo simples (Modelo \textit{Naive}) de previsão e uma técnica simples de operação na bolsa de valores (\textit{Naive Trading}). A rede LSTM apresentou um desempenho muito semelhante aos modelos ingênuos, com prejuízo em relação ao valor investido inicialmente. Em contrapartida, a rede CNN teve um desempenho de +73\% de acerto de operações com lucro em relação ao Modelo \textit{Naive}, bem como +20\% de lucro frente ao algoritmo Naive Trading. Ambas as redes (CNN e LSTM) tiveram desempenho geral inferior ao modelo AR. Entretanto, é notável que a acurácia média de operações com lucro do CNN foi de 59\%, valor muito próximo ao do modelo AR, de 60\%, levantando o questionamento a respeito da rede em diferentes ações para estabelecer uma base comparativa mais ampla em relação ao modelo AR.

\par
Acerca das características específicas de cada rede, bem como potenciais situações de uso de cada uma frente a determinados problemas específicos, a CNN tem uma adaptação boa para a previsão que visa classificar valor em séries temporais (por exemplo em “alta” ou “baixa” para o movimento da bolsa de valores), visto que teve uma acurácia de operação muito próxima do modelo AR, que é robusto para problemas desta natureza. Já a rede LSTM, apesar de não ter apresentado grandes lucros para esse algoritmo em específico, treinada sob estas condições para a ação PETR4.SA, apresentou uma aproximação maior que o CNN para os valores numéricos em si; as previsões, ainda que classificadas incorretamente na maior parte dos casos, tiveram uma proximidade numérica maior com os valores reais de fechamento.

\par
Por fim, a rede CNN apresentou um desempenho majoritariamente bom para a classificação dos movimentos previsão numérica e operação de trading; o modelo AR um desempenho ótimo; a rede LSTM um desempenho ruim (para os experimentos deste relatório); e o modelo \textit{Naive}, um desempenho ruim também.