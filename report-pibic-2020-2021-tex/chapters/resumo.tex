% resumo em português
\setlength{\absparsep}{18pt} % ajusta o espaçamento dos parágrafos do resumo
\begin{resumo}
\addcontentsline{toc}{section}{RESUMO}
A previsão de movimentos no mercado de ações é uma tarefa complexa, apresentando inúmeras abordagens que objetivam uma maximização dos lucros em contrapartida de uma minimização das perdas e dos riscos por parte dos investidores. As técnicas mais recentes de redes neurais (CNN e LSTM), apesar de seu ótimo desempenho em inúmeras áreas, contrastam com a ausência de trabalhos voltados para sua utilização no mercado de ações. Este trabalho buscou comparar tais técnicas com o modelo Autorregressivo, clássico na previsão em séries temporais, bem como a modelos mais simples e ``ingênuos'' de previsão, de maneira a ter uma base maior de discussão. Os modelos foram testados com as ações PETR4.SA (Petrobrás) para os anos 2019 e 2020, oscilando em períodos diferentes de treino e teste, bem como repetindo os experimentos para uma maior assertividade dos resultados. As previsões geradas pelos modelos foram ainda submetidas à operações de \textit{trading} nos períodos de teste, utilizando classificações de ``alta'' e ``baixa'' em relação ao preço de fechamento do dia seguinte na tomada de decisão. Os rótulos dos movimentos foram gerados através de um algoritmo específico que separava em uma das duas classes os preços numéricos obtidos pelos modelos. Os resultados apontaram que a técnica CNN apresentou um desempenho aproximado em relação ao modelo Autorregressivo em termos de acurácia de operações com lucro (60\%), bem como +73\% desta acurácia em relação ao \textit{Naive Model} (Modelo Ingênuo) e +20\% em relação ao algoritmo de investimento \textit{Naive Trading} (Trading Ingênuo, utilizando apenas os preços dos dois dias anteriores para a tomada de decisão). Os experimentos realizados nas condições específicas deste trabalho, em termos dos parâmetros utilizados e o período da ação em questão, demostraram que o LSTM teve um desempenho ineficaz no que diz respeito às classificações dos movimentos e operação de \textit{trading}, bem como um desempenho comparável ao modelo autorregressivo em termos de previsões numéricas, considerando a proximidade como o valor real de fechamento. A técnica CNN foi especialmente eficiente para os propósitos deste trabalho, tal como é ressaltado na bibliografia consultada em relação ao seu uso para predição em séries temporais.

 \textbf{Palavras-chave}: \textit{Deep Learning}, CNN, LSTM, Mercado de Ações, Séries Temporais.
 
\end{resumo}